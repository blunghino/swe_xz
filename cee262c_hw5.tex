\documentclass[12pt]{article}

% for lettered subsections
\renewcommand\thesubsection{\alph{subsection}}
% for roman numeralled subsubsections
\renewcommand\thesubsubsection{\roman{subsubsection}}

% for margins
\usepackage[margin=0.75in]{geometry}

% for figures...
\usepackage{graphicx}

% for split environment and cases
\usepackage{amsmath} 
\usepackage{amsfonts}

% for \FloatBarrier to keep figs in sections
\usepackage[section]{placeins}

% color text
\usepackage{color}

% for tables
\usepackage{tabularx, caption}

% for tilde 
\usepackage{textcomp}
\newcommand{\textapprox}{\raisebox{0.5ex}{\texttildelow}}

% scientific notation, units
\usepackage{siunitx}

\title{CEE262c HW5}  
\author{Brent Lunghino\\lunghino@stanford.edu}

\begin{document}
    
\maketitle

In this assignment we work with the 2D-V linearized shallow water equations:

\begin{equation} \label{eq:swe_x_momentum}
\frac{\partial u}{\partial t} = -g \frac{\partial h}{\partial x} - \frac{\partial q}{\partial x}
\end{equation}

\begin{equation} \label{eq:swe_z_momentum}
\frac{\partial w}{\partial t} = -\frac{\partial q}{\partial z}
\end{equation}

\begin{equation} \label{eq:swe_continuity}
\frac{\partial h}{\partial t} = -\frac{\partial}{\partial x} \int_{-D}^{0} u~ dz
\end{equation}

% q1
\section{}

Each term in (\ref{eq:swe_x_momentum}) is discretized as follows:

\begin{equation*}
\frac{\partial u}{\partial t} = \frac{u_{i,k}^{n+1} - u_{i,k}^*}{\Delta t} + \frac{u_{i,k}^* - u_{i,k}^n}{\Delta t}
\end{equation*}

\begin{equation*}
-g \frac{\partial h}{\partial x} = - \frac{g \theta}{\Delta x} (h_{i}^{n+1}-h_{i-1}^{n+1}) - \frac{g (1-\theta)}{\Delta x} (h_{i}^n - h_{i-1}^n)
\end{equation*} 

\begin{equation*}
- \frac{\partial q}{\partial x} = -\frac{q_{i,k}^c - q_{i-1,k}^c}{\Delta x} - \frac{q_{i,k}^{n-1/2} - q_{i-1,k}^{n-1/2}}{\Delta x}
\end{equation*}

Where the pressure field is to be updated as:

\begin{equation*} \label{eq:pressure_update}
q_{i,k}^{n+1/2} = q_{i,k}^c + q_{i,k}^{n-1/2}
\end{equation*}

And define $u^*$ such that:

\begin{equation} \label{eq:u-velocity_pressure_correction}
\frac{u_{i,k}^{n+1} - u_{i,k}^*}{\Delta t} = -\frac{q_{i,k}^c - q_{i-1,k}^c}{\Delta x} 
\end{equation}

\begin{equation} \label{eq:u-velocity_update}
u_{i,k}^{n+1} = u_{i,k}^* - \frac{\Delta t}{\Delta x} (q_{i,k}^c - q_{i-1,k}^c)
\end{equation}

Using the above discretizations in (\ref{eq:swe_x_momentum}) and subtracting (\ref{eq:u-velocity_pressure_correction}):

\begin{equation*}
\frac{u_{i,k}^* - u_{i,k}^n}{\Delta t} = - \frac{g \theta}{\Delta x} (h_{i}^{n+1}-h_{i-1}^{n+1}) - \frac{g (1-\theta)}{\Delta x} (h_{i}^n - h_{i-1}^n) - \frac{1}{\Delta x} (q_{i,k}^{n-1/2} - q_{i-1,k}^{n-1/2})
\end{equation*}

\begin{equation} \label{eq:u_predictor_velocity}
u_{i,k}^* = S_{i,k} - \frac{g \theta \Delta t}{\Delta x} (h_{i}^{n+1} - h_{i-1}^{n+1})
\end{equation}

In (\ref{eq:u_predictor_velocity}), $S_{i,k}$ is given by:

\begin{equation*}
S_{i,k} = u_{i,k}^n - \frac{g \Delta t (1-\theta)}{\Delta x} (h_{i}^n - h_{i-1}^n) - \frac{\Delta t}{\Delta x} (q_{i,k}^{n-1/2} - q_{i-1,k}^{n-1/2})
\end{equation*}

To discretize (\ref{eq:swe_z_momentum}) we use the same pressure update given by (\ref{eq:pressure_update}) and discretize each term as follows:

\begin{equation*}
\frac{\partial w}{\partial t} = \frac{w_{i,k}^{n+1} - w_{i,k}^*}{\Delta t} + \frac{w_{i,k}^* - w_{i,k}^n}{\Delta t}
\end{equation*}

\begin{equation*}
- \frac{\partial q}{\partial z} = -\frac{q_{i,k}^c - q_{i,k-1}^c}{\Delta z} - \frac{q_{i,k}^{n-1/2} - q_{i,k-1}^{n-1/2}}{\Delta z}
\end{equation*}

Define $w^*$ such that:

\begin{equation} \label{eq:w-velocity_pressure_correction}
\frac{w_{i,k}^{n+1} - w_{i,k}^*}{\Delta t} = -\frac{q_{i,k}^c - q_{i,k-1}^c}{\Delta z} 
\end{equation}

Plugging the above discretizations into (\ref{eq:swe_z_momentum}) and subtracting (\ref{eq:w-velocity_pressure_correction}):

\begin{equation*}
\frac{w_{i,k}^* - w_{i,k}^n}{\Delta t} = - \frac{q_{i,k}^{n-1/2} - q_{i,k-1}^{n-1/2}}{\Delta z}
\end{equation*}

\begin{equation} \label{eq:w_predictor_velocity}
w_{i,k}^* = w_{i,k}^n - \frac{\Delta t}{\Delta z} (q_{i,k}^{n-1/2} - q_{i,k-1}^{n-1/2})
\end{equation}

The u-velocity field is bounded by walls on both sides such that at all timesteps $u_{0,k} = u_{N_i,k} = 0$. The w-velocity field is bounded by a wall on the bottom such that for all timesteps $w_{i,0} = 0$. At the free surface we must assume $q^c = 0$ such that $q_{i,N_k} = -q_{i,N_k-1}$. This allows (\ref{eq:w_predictor_velocity}) to be solved at the free surface:

\begin{equation*}
w_{i,N_k}^* = w_{i,N_k}^n - \frac{\Delta t}{\Delta z} (q_{i,N_k}^{n-1/2} - q_{i,N_k-1}^{n-1/2})
\end{equation*}

\begin{equation} \label{eq:w_str_free_surface}
w_{i,N_k}^* = w_{i,N_k}^n + \frac{2\Delta t}{\Delta z} q_{i,N_k-1}^{n-1/2}
\end{equation}

In equation (\ref{eq:swe_continuity}) we discretize each term as follows:

\begin{equation*}
\frac{\partial h}{\partial t} = \frac{h_{i}^{n+1} - h_i^n}{\Delta t}
\end{equation*}

\begin{align*}
-\frac{\partial}{\partial x} \int_{-D}^{0} u~ dz &= - \theta  \frac{\partial}{\partial x} \int_{-D}^{0} u^*~ dz - (1-\theta) \frac{\partial}{\partial x} \int_{-D}^{0} u^n~ dz
\\
 &= - \theta  \frac{\partial}{\partial x} \sum_{k=1}^{N_k} u_{i,k}^* \Delta z - (1-\theta) \frac{\partial}{\partial x} \sum_{k=1}^{N_k} u_{i,k}^n \Delta z
\\
 &= - \frac{\theta \Delta z}{\Delta x} \sum_{k=1}^{N_k} (u_{i+1,k}^* - u_{i,k}^*) -  \frac{(1-\theta) \Delta z}{\Delta x} \sum_{k=1}^{N_k} (u_{i+1,k}^n - u_{i,k}^n)
\end{align*}

Note in the above expression we have assumed that $u^{n+1} - u^*$ is small and replaced $u^{n+1}$ by $u^*$ in the theta method. Equation (\ref{eq:swe_continuity}) becomes:

\begin{equation} \label{eq:free_surface_theta}
\frac{h_{i}^{n+1} - h_i^n}{\Delta t} = - \frac{\theta \Delta z}{\Delta x} \sum_{k=1}^{N_k} (u_{i+1,k}^* - u_{i,k}^*) -  \frac{(1-\theta) \Delta z}{\Delta x} \sum_{k=1}^{N_k} (u_{i+1,k}^n - u_{i,k}^n)
\end{equation}

\begin{equation} \label{eq:h_update}
h_{i}^{n+1} = h_i^n - \frac{\theta \Delta t \Delta z}{\Delta x} \sum_{k=1}^{N_k} (u_{i+1,k}^* - u_{i,k}^*) -  \frac{(1-\theta) \Delta t \Delta z}{\Delta x} \sum_{k=1}^{N_k} (u_{i+1,k}^n - u_{i,k}^n)
\end{equation}

Substituting in (\ref{eq:u_predictor_velocity}) for $u^*$:

\begin{multline*}
h_{i}^{n+1} = h_i^n  - \frac{(1-\theta) \Delta t \Delta z}{\Delta x} \sum_{k=1}^{N_k} (u_{i+1,k}^n - u_{i,k}^n) ...
\\ 
- \frac{\theta \Delta t \Delta z}{\Delta x} \sum_{k=1}^{N_k} \left( S_{i+1,k} - \frac{g \theta \Delta t}{\Delta x} (h_{i+1}^{n+1} - h_{i}^{n+1}) - S_{i,k} + \frac{g \theta \Delta t}{\Delta x} (h_{i}^{n+1} - h_{i-1}^{n+1}) \right)
\end{multline*}

\begin{multline*}
h_{i}^{n+1} - \frac{\theta \Delta t D}{\Delta x} \left(\frac{g \theta \Delta t}{\Delta x} (h_{i+1}^{n+1} - h_{i}^{n+1}) - \frac{g \theta \Delta t}{\Delta x} (h_{i}^{n+1} - h_{i-1}^{n+1}) \right) ... 
\\= h_i^n  - \frac{(1-\theta) \Delta t \Delta z}{\Delta x} \sum_{k=1}^{N_k} (u_{i+1,k}^n - u_{i,k}^n) - \frac{\theta \Delta t \Delta z}{\Delta x} \sum_{k=1}^{N_k} ( S_{i+1,k} - S_{i,k} )
\end{multline*}

\begin{equation*}
-a h_{i-1}^{n+1} + (1+2a) h_i^{n+1} -a h_{i+1}^{n+1} = R_i
\end{equation*}

Where:

\begin{equation*}
a = \frac{\theta^2 g D \Delta t^2}{\Delta x^2}
\end{equation*}

\begin{equation*}
R_i = h_i^n  - \frac{(1-\theta) \Delta t \Delta z}{\Delta x} \sum_{k=1}^{N_k} (u_{i+1,k}^n - u_{i,k}^n) - \frac{\theta \Delta t \Delta z}{\Delta x} \sum_{k=1}^{N_k} ( S_{i+1,k} - S_{i,k} )
\end{equation*}

At the boundary cells we use the no-flux boundary condition to complete the system of equations. Starting from (\ref{eq:h_update}) we derive the equations for $h_0$ and $h_{N_i-1}$:

\begin{align*}
h_0^{n+1} &= h_0^n  -  \frac{(1-\theta) \Delta t \Delta z}{\Delta x} \sum_{k=1}^{N_k} (u_{1,k}^n - u_{0,k}^n) - \frac{\theta \Delta t \Delta z}{\Delta x} \sum_{k=1}^{N_k} (u_{1,k}^* - u_{0,k}^*)
\\
h_0^{n+1} &= h_0^n  -  \frac{(1-\theta) \Delta t \Delta z}{\Delta x} \sum_{k=1}^{N_k} (u_{1,k}^n) - \frac{\theta \Delta t \Delta z}{\Delta x} \sum_{k=1}^{N_k} (S_{1,k} - \frac{g \theta \Delta t}{\Delta x} (h_1^{n+1} - h_0^{n+1}))
\\
(1+a)h_0^{n+1} - ah_1^{n+1} &=  h_0^n  -  \frac{(1-\theta) \Delta t \Delta z}{\Delta x} \sum_{k=1}^{N_k} (u_{1,k}^n) - \frac{\theta \Delta t \Delta z}{\Delta x} \sum_{k=1}^{N_k} (S_{1,k} ) = R_0
\end{align*}
\begin{align*}
h_{N_i-1}^{n+1} &= h_{N_i-1}^n  -  \frac{(1-\theta) \Delta t \Delta z}{\Delta x} \sum_{k=1}^{N_k} (u_{N_i,k}^n - u_{N_i-1,k}^n) - \frac{\theta \Delta t \Delta z}{\Delta x} \sum_{k=1}^{N_k} (u_{N_i,k}^* - u_{N_i-1,k}^*)
\\
h_{N_i-1}^{n+1} &= h_{N_i-1}^n  +  \frac{(1-\theta) \Delta t \Delta z}{\Delta x} \sum_{k=1}^{N_k} ( u_{N_i-1,k}^n) + \frac{\theta \Delta t \Delta z}{\Delta x} \sum_{k=1}^{N_k} (S_{N_i-1,k} - \frac{g \theta \Delta t}{\Delta x} (h_{N_i-1}^{n+1} - h_{i-2}^{n+1}))
\\
(1+a)h_{N_i-1}^{n+1} -ah_{N_i-2}^{n+1} &= h_{N_i-1}^n  +  \frac{(1-\theta) \Delta t \Delta z}{\Delta x} \sum_{k=1}^{N_k} ( u_{N_i-1,k}^n) + \frac{\theta \Delta t \Delta z}{\Delta x} \sum_{k=1}^{N_k} (S_{N_i-1,k}) = R_{N_i-1}
\end{align*}

To solve for $h^{n+1}$ solve the following tridiagonal system:

\begin{equation*}
\begin{pmatrix} 
(1 + a) & -a 
\\  & \ddots & \ddots 
\\ & & -a & (1 + 2a) & -a
\\ & & & \ddots & \ddots
\\ & & & & -a & (1 + a)
\end{pmatrix} \begin{pmatrix} ~\\~ \\ h \\ ~\\~ \end{pmatrix}^{n+1} = \begin{pmatrix} R_0 \\ \vdots \\ R_j \\ \vdots \\ R_{N_i-1} \end{pmatrix}
\end{equation*}

% q2
\section{}

In (\ref{eq:u-velocity_update}) and (\ref{eq:w-velocity_pressure_correction}) we dicretely defined $\vec{u}^{n+1} = \vec{u}^* - \Delta t \vec{\nabla} q^c$. Taking the divergence and using $\vec{\nabla} \cdot \vec{u}^{n+1} = 0$:

\begin{align*}
\vec{u}^{n+1} &= \vec{u}^* - \Delta t \vec{\nabla} q^c
\\
\nabla^2 q^c &= \frac{1}{\Delta t} \vec{\nabla} \cdot \vec{u}^*
\end{align*}

This can be written discretely as:

\begin{equation} \label{eq:poisson_with_derivatives}
\frac{1}{\Delta x} \left( \frac{\partial q^c}{\partial x} \bigg|_{i+1,k} - \frac{\partial q^c}{\partial x} \bigg|_{i,k} \right) 
+ \frac{1}{\Delta z} \left( \frac{\partial q^c}{\partial z} \bigg|_{i,k+1} - \frac{\partial q^c}{\partial z} \bigg|_{i,k} \right) = b_{i,k}
\end{equation}

\begin{equation} \label{eq:discrete_poisson}
\frac{1}{\Delta x^2} (q_{i+1,k}^c - 2 q_{i,k}^c + q_{i-1,k}^c) 
+ \frac{1}{\Delta z^2} (q_{i,k+1}^c- 2 q_{i,k}^c + q_{i,k-1}^c) = b_{i,k}
\end{equation}

\begin{equation*}
b_{i,k} = \frac{1}{\Delta t} \left( \frac{u_{i+1,k}^* - u_{i,k}^*}{\Delta x}  + \frac{w_{i,k+1}^* - w_{i,k}^*}{\Delta z}\right)
\end{equation*}

Using the no-flux condition at the walls ($u_{0,k}=u_{N_i,k}=0$ and $w_{i,0} = 0$) in (\ref{eq:u-velocity_pressure_correction}):

\begin{align*}
\frac{u_{i=\text{wall},k}^{n+1} - u_{i=\text{wall},k}^*}{\Delta t} &= -\frac{q_{i,k}^c - q_{i-1,k}^c}{\Delta x}
\\
0 &= \frac{\partial q^c}{\partial x} \bigg|_{\text{wall},k}
\end{align*}

The result is analagous in (\ref{eq:w-velocity_pressure_correction}). With this information and with the free surface condition $q_{i,N_k}=-q_{i,N_k-1}$ as in (\ref{eq:w_str_free_surface}) we can rewrite (\ref{eq:discrete_poisson}) to work at the 8 different types of boundaries:

At $(0,k)$:
\begin{equation*}
\frac{1}{\Delta x^2} (q_{1,k}^c - q_{0,k}^c) 
+ \frac{1}{\Delta z^2} (q_{0,k+1}^c- 2 q_{0,k}^c + q_{0,k-1}^c) = \frac{1}{\Delta t} \left( \frac{u_{1,k}^*}{\Delta x}  + \frac{w_{0,k+1}^* - w_{0,k}^*}{\Delta z}\right)
\end{equation*}

At $(0,0)$:
\begin{equation*}
\frac{1}{\Delta x^2} (q_{1,0}^c -  q_{0,0}^c) 
+ \frac{1}{\Delta z^2} (q_{0,1}^c-  q_{0,0}^c) = \frac{1}{\Delta t} \left( \frac{u_{1,0}^*}{\Delta x}  + \frac{w_{0,1}^*}{\Delta z}\right)
\end{equation*}

At $(i,0)$:
\begin{equation*}
\frac{1}{\Delta x^2} (q_{i+1,0}^c - 2 q_{i,0}^c + q_{i-1,0}^c) 
+ \frac{1}{\Delta z^2} (q_{i,1}^c- q_{i,0}^c) = \frac{1}{\Delta t} \left( \frac{u_{i+1,0}^* - u_{i,0}^*}{\Delta x}  + \frac{w_{i,1}^*}{\Delta z}\right)
\end{equation*}

At $(N_i-1,0)$:
\begin{equation*}
\frac{1}{\Delta x^2} ( -  q_{N_i-1,0}^c + q_{N_i-2,0}^c) 
+ \frac{1}{\Delta z^2} (q_{N_i-1,1}^c-  q_{N_i-1,0}^c) = \frac{1}{\Delta t} \left( \frac{- u_{N_i-1,0}^*}{\Delta x}  + \frac{w_{N_i-1,1}^* }{\Delta z}\right)
\end{equation*}

At $(N_i-1,k)$:
\begin{equation*}
\frac{1}{\Delta x^2} (- 2 q_{N_i-1,k}^c + q_{N_i-2,k}^c) 
+ \frac{1}{\Delta z^2} (q_{N_i-1,k+1}^c- 2 q_{N_i-1,k}^c + q_{N_i-1,k-1}^c) = \frac{1}{\Delta t} \left( \frac{ - u_{N_i-1,k}^*}{\Delta x}  + \frac{w_{N_i-1,k+1}^* - w_{N_i-1,k}^*}{\Delta z}\right)
\end{equation*}

At $(N_i-1, N_k-1)$:
\begin{equation*}
\frac{1}{\Delta x^2} ( -  q_{N_i-1, N_k-1}^c + q_{N_i-2, N_k-1}^c) 
+ \frac{1}{\Delta z^2} (- 3 q_{N_i-1, N_k-1}^c + q_{N_i-1, N_k-2}^c) = \frac{1}{\Delta t} \left( \frac{ - u_{N_i-1, N_k-1}^*}{\Delta x}  + \frac{w_{i,k+1}^* - w_{i,k}^*}{\Delta z}\right)
\end{equation*}

At $(i,N_k-1)$:
\begin{equation*}
\frac{1}{\Delta x^2} (q_{i+1,N_k-1}^c -  2q_{i,N_k-1}^c + q_{i-1,N_k-1}^c) 
+ \frac{1}{\Delta z^2} (- 3 q_{i,N_k-1}^c + q_{i,N_k-2}^c) = \frac{1}{\Delta t} \left( \frac{u_{i+1,N_k-1}^* - u_{i,N_k-1}^*}{\Delta x}  + \frac{w_{i,N_k}^* - w_{i,N_k-1}^*}{\Delta z}\right)
\end{equation*}

At $(0,N_k-1)$:
\begin{equation*}
\frac{1}{\Delta x^2} (q_{1,N_k-1}^c -  q_{0,N_k-1}^c ) 
+ \frac{1}{\Delta z^2} (-3 q_{0,N_k-1}^c + q_{0,N_k-2}^c) = \frac{1}{\Delta t} \left( \frac{u_{1,N_k-1}^* }{\Delta x}  + \frac{w_{i,k+1}^* - w_{i,k}^*}{\Delta z}\right)
\end{equation*}

Using these boundary conditions and (\ref{eq:discrete_poisson}) a pentadiagonal matrix can be constructed and $q^c$ can be computed using a conjugate gradient solver.

After solving the system of equations for $q^c$ use (\ref{eq:pressure_update}) to calculate $q^{n+1/2}$. Then use (\ref{eq:u-velocity_update}) to update the u-velocity field $u^{n+1}$. To update the w-velocity field use the fact that the flow is divergence free:

\begin{equation*}
\frac{w_{i,k+1}^{n+1} - w_{i,k}^{n+1}}{\Delta z} = - \frac{u_{i+1,k}^{n+1} - u_{i,k}^{n+1}}{\Delta x}
\end{equation*}

\begin{equation*}
w_{i,k+1}^{n+1} = w_{i,k}^{n+1} - \frac{\Delta z}{\Delta x} (u_{i+1,k}^{n+1} - u_{i,k}^{n+1})
\end{equation*}

Starting with the bottom boundary condition $w_{i,0} = 0$ this system of equations can be solved by back substitution.

% q3
\section{}

Figure \ref{fig:timeseries_freesurface} shows the time evolution of the free surface comparing the analytical result to the non-hydrostatic and hydrostatic numerical results. The comparison is considered for three domains with varying depth to length ratios: $D/L \in \{1, 1/4, 1/8\}$. For $D/L = 1$ the non-hydrostatic model result is out of phase with the analytical. The wavelength from the hydrostatic model is much shorter than the analytical. As $D/L$ decreases, the wavelength from the hydrostatic model increases and the non-hydrostatic phase shift decreases. For $D/L = 8$ the hydrostatic and non-hydrostatic model results both nearly match the analytical. The results of the hydrostatic model in particular are significantly improved because neglecting the non-hydrostatic pressure is a more appropriate assumption for the lower $D/L$ ratio, in which vertical velocities are small relative to horizontal.

	\begin{figure}[htb]
		\begin{center}
			\includegraphics[width=\linewidth]{fig/timeseries_freesurface.png}
			\caption{Timeseries of free surface height normalized by initial condition amplitude at $x/L=0$. Comparison of theoretical, non-hydrostatic and hydrostatic results for $D/L \in \{1, 1/4, 1/8\}$.}
			\label{fig:timeseries_freesurface}
		\end{center}
	\end{figure}
	
Figure \ref{fig:snapshot_velocity_profiles} compares the vertical velocity profiles for the hydrostatic and non-hydrostatic models for the same three $D/L$ ratios. The hydrostatic profiles remain linear for all $D/L$ ratios. The u velocity is roughly constant with depth and the w velocity decreases from 1 at the surface to zero at the bottom. The non-hydrostatic profiles change significantly as $D/L$ varies. For $D/L = 1$ the velocity profiles for both u and w decrease non-linearly, with approximately 50\% of the velocity decrease occuring in the top 20\% of the water column. As $D/L$ is decreased the velocity profiles converge to roughly match the linear hydrostatic profiles (the w velocity converges first). This occurs because for lower $D/L$ the hydrostatic assumption is a more appropriate model for the physical characteristics of the problem.

	\begin{figure}[htb]
		\begin{center}
			\includegraphics[width=\linewidth]{fig/snapshot_velocity_profiles.png}
			\caption{Vertical profiles of velocity for $u$ and $w$ normalized by their maximum value along the profile. Comparison of non-hydrostatic and hydrostatic velocity profiles for $D/L \in \{1, 1/4, 1/8\}$.}
			\label{fig:snapshot_velocity_profiles}
		\end{center}
	\end{figure}

% q4
\section{}

With $k = 2\pi/\lambda$ and $\lambda = 2L$, the theoretical wave phase speed is:

\begin{equation} \label{eq:dispersion_relation}
c = \sqrt{\frac{g}{k} \tanh(kD)}
\end{equation}

The deep-water limit of the dispersion relation is:

\begin{equation} \label{eq:deep-water_limit}
c_0 = \sqrt{\frac{g}{k}}
\end{equation}

The shallow-water limit is:

\begin{equation} \label{eq:shallow-water_limit}
c_s = \sqrt{gD}
\end{equation}

	\begin{figure}[htb]
		\begin{center}
			\includegraphics[width=0.8\linewidth]{fig/plot_dispersion_relations.png}
			\caption{Theoretical and numerical dispersion relations for wave phase speed normalized by the deep-water limit. The theoretical intermediate value is given by (\ref{eq:dispersion_relation}). The theoretical deep-water limit is given by (\ref{eq:deep-water_limit}) and the theoretical shallow-water limit is given by (\ref{eq:shallow-water_limit}).}
			\label{fig:plot_dispersion_relations}
		\end{center}
	\end{figure}

% q5
\section{}

Figure \ref{fig:timeseries_freesurface_projection_correction_20} ($\Delta t = T/20$) and Figure \ref{fig:timeseries_freesurface_projection_correction_80} ($\Delta t = T/80$) compare the freesurface evolution of the projection and correction methods for the non-hydrostatic pressure field. The projection method has a lot of diffusion, for $\Delta t = T/20$ the diffusion takes the free surface fluctuations to essentially zero by about 5 wave periods. For $\Delta t = T/20$ the correction method also shows significant diffusion, this comes from the assumption in (\ref{eq:free_surface_theta}) that $u^{n+1}$ can be replaced by $u^*$. The correction method also suffers from dispersion, which is expected. For $\Delta t = T/20$ the numerical dispersion causes the phase of the free surface cosine wave shifts by over a half period over 10 wave periods. For $\Delta t = T/80$ the phase of the free surface cosine wave shifts by only a small fraction of a period.

	\begin{figure}[htb]
		\begin{center}
			\includegraphics[width=0.85\linewidth]{fig/timeseries_freesurface_projection_correction_20.png}
			\caption{$\Delta t = T/20$. Timeseries of free surface height normalized by initial condition amplitude at $x/L=0$. Comparison of analytical solution to projection and correction methods for non-hydrostatic pressure calculation.}
			\label{fig:timeseries_freesurface_projection_correction_20}
		\end{center}
	\end{figure}
	
	\begin{figure}[htb]
		\begin{center}
			\includegraphics[width=0.85\linewidth]{fig/timeseries_freesurface_projection_correction_80.png}
			\caption{$\Delta t = T/80$. Timeseries of free surface height normalized by initial condition amplitude at $x/L=0$. Comparison of analytical solution to projection and correction methods for non-hydrostatic pressure calculation.}
			\label{fig:timeseries_freesurface_projection_correction_80}
		\end{center}
	\end{figure}
\FloatBarrier

Figures (\ref{fig:plot_accuracy_correction}) and (\ref{fig:plot_accuracy_projection}) show the time accuracy of the correction and projection methods respectively. Error is computed as $(||f(\Delta t) - f(\Delta t / 2)||_2 )_{t=t_{\text{max}}}$, the relative error. The correction method is second order accurate in time. The projection method is first order accurate in time.

	\begin{figure}[htb]
		\begin{center}
			\includegraphics[width=0.65\linewidth]{fig/plot_accuracy_correction.png}
			\caption{Time accuracy of the correction method for non-hydrostatic pressure calculation}
			\label{fig:plot_accuracy_correction}
		\end{center}
	\end{figure}
	
	\begin{figure}[htb]
		\begin{center}
			\includegraphics[width=0.65\linewidth]{fig/plot_accuracy_projection.png}
			\caption{Time accuracy of the projection method for non-hydrostatic pressure calculation}
			\label{fig:plot_accuracy_projection}
		\end{center}
	\end{figure}

\end{document}